\label{chap:conclusion}
High performance systems are evolving to the point where performance is no longer the sole relevant criterion anymore. This thesis makes an argument that current execution and resource management paradigms will no longer be sufficient to ensure performance or to meet goals. Power requirements are rapidly driving the co-design of HPC systems, which in turn has set the course for a radical shift in how to express the need for scarcer and scarcer resources, as well as manage them. This thesis opened with a discussion of why systems will need to become more introspective and self-aware with respect to performance, energy, and resiliency.

To this end, we explored the background and design of a Target Exascale Architecture based off of current trends in HPC, as well as, the experimental design of new exascale architectures. Then, we discussed relevant hardware and software metrics for adaptive exascale systems; as well as, the challenges and opportunities within self-aware systems. Following this, we focused on formulating the problem of self-aware systems in concrete terms, and discussed hardware/software requirements to enable adaptation. Next, we detailed, SAFE, a framework and simulator for experimenting with distributed self-adaptive control policies. Following this, we provided an experimental evaluation of adaptive control policies ranging from localized to hierarchical adaptive schemes under various conditions. Through this, we demonstrated the need for hierarchical adaptive control mechanisms, as well as, characterized the effects of maladaptive conditions on adaptive policies. Finally, we characterized the current field of adaptive computing focusing on different types of adaption from application centric, component centric, to system centric designs.  We additionally looked at cross-layer adaptation (as focused on in this thesis), modeling, and the current state of the art in self-adaptive systems. 

Through the discussion and experimentation, several important conclusions were reached. These conclusions are stated succinctly as follows: (1) Distributed control yields more risks than traditional control systems. (2) Bad actors give rise to unpredictable oscillation and this behavior is demonstrably mitigated by hierarchical distributed adaptive solutions. (3) Delayed control is particularly maladaptive leading toward unstable oscillations even for stable workloads. (4) The correctness and timely movement of observation data is critically important for distributed adaptation. (4) Quick decisions are needed otherwise reactive decision making is not possible within a distributed system. (5) Localized adaptation is not enough to converge to system level goals because it may not be possible for localized control engines to converge toward a goal. (6) Aggregation of information is critical to mitigate the effect of maladaptive components and a proactive controller is needed to identify such cases and quell them. (6) Fine-grained control over system resources is needed for adaptive policies, without which control engines may be unable to meet their goals.

These conclusions necessarily require that hardware support the infrastructure to monitor system state, as well as, provide the capability to quickly move data among sections of the system. Moreover, granular control over system state is needed to ease the the burden of adaptive control by enabling localized convergence to goals not otherwise possible. In essence, the development and design of hardware supporting these features is a hard requirement if exascale systems are to become more self-aware and handle the variability of complex workloads spanning thousands of cores. As we draw nearer to rise of exascale, we will begin to see the issues detailed in thesis become more prevalent. Research and development will need to yield better and more optimized system software capable of efficient scheduling and introspective behavior. And necessarily, hardware will begin to reflect these needs.

Though only touched upon briefly in this thesis, resilience and fault tolerance will also become large issues moving into the exascale era due to the variable of component yield, increased transistor counts due to submicron scaling~\cite{Jouppi2009}, and the complexity of massively scaled chip designs; as well as, the move to near threshold voltage (NTV) operation~\cite{AmarasingheEtAl2011,Borkar2011}. In such systems, introspective system software capable of adaptive and predictive behavior will be needed to manage all of these new aspects of exascale computing, and new bodies of research will need be developed in order to meet these new demands.
